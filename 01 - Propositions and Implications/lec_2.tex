\section{Variables and Sets}

\subsection*{Predicates and the Need for Abstraction}
Writing a separate symbol for every single statement,
such as  
$P$: “2 is prime”, $Q$: “3 is prime”, $R$: “4 is prime”,  
quickly becomes unmanageable when the numbers grow.
A better idea is to allow a statement to depend on a \emph{variable}.

\begin{definition}[Predicate]
A \emph{predicate} is a sentence containing one or more variables
that becomes a proposition once the variables are given specific values.
\end{definition}

For example, define
\[
P(x):\; “x \text{ is prime }.”
\]
Then $P(7)$ is true, $P(10)$ is false, and $P(a+b)$ expresses  
“$a+b$ is prime,” which depends on the values of $a$ and $b$.

\begin{example}
Let $D(x,y)$ mean “$x$ is divisible by $y$.”  
Then $D(9,3)$ is true because $3$ divides $9$,  
while $D(10,3)$ is false.
\end{example}

\subsection*{Sets and Membership}
A \emph{set} is a collection of objects, called \emph{elements},  
that we wish to consider as a single whole.
Order and repetition do not matter:
\[
\{1,2,5,25\} = \{25,1,2,5\}.
\]
We write $x\in S$ for “$x$ is an element of $S$” and $x\notin S$
if $x$ is not in $S$.

\begin{example}
Define
\[
P = \{x \mid x \text{ is prime}\}.
\]
Because $7$ is prime, $7\in P$, whereas $8\notin P$.
\end{example}

\begin{lemma}[Renaming Bound Variables]
If $S=\{x\mid \varphi(x)\}$ and $T=\{y\mid \varphi(y)\}$ for the same predicate $\varphi$,
then $S=T$.
\end{lemma}

\begin{proof}
Let $a\in S$.  
Then $\varphi(a)$ holds, so $a\in T$.  
Similarly every element of $T$ is in $S$, proving equality.
The choice of bound-variable name is irrelevant.
\end{proof}

\subsection*{Free vs.\ Bound Variables}
In the expression $x^2<25$, the variable $x$ is \emph{free}:
its name matters and it represents a specific unknown number.
In contrast, in $\{x\mid x^2<25\}$ the symbol $x$ is \emph{bound}:
we may replace it with $y$ or $t$ without changing the set.

\subsection*{Truth Sets and Logical Connections}
\begin{definition}[Truth Set]
Given a predicate $P(x)$ and a universe $U$,  
the \emph{truth set} of $P$ is
\[
\{x\in U \mid P(x)\text{ is true}\}.
\]
\end{definition}

\begin{example}
Within $\mathbb R$ the truth set of $x^2-4x+3=0$ is $\{1,3\}$,  
since those are the real roots.  
The truth set of $x^2+3=0$ is $\emptyset$ because the equation has no real solutions.
\end{example}

Logical connectives correspond to set operations.  
If $A$ is the truth set of $P(x)$ and $B$ of $Q(x)$:
\[
\begin{aligned}
\text{truth set of } P(x)\land Q(x) &= A\cap B,\\
\text{truth set of } P(x)\lor Q(x)  &= A\cup B,\\
\text{truth set of } \lnot P(x)     &= U\setminus A.
\end{aligned}
\]

\subsection*{Basic Set Operations}
\begin{align*}
A\cap B &= \{x\mid x\in A \text{ and } x\in B\} && \text{(intersection)},\\
A\cup B &= \{x\mid x\in A \text{ or } x\in B\}  && \text{(union)},\\
A\setminus B &= \{x\mid x\in A \text{ and } x\notin B\} && \text{(difference)},\\
\overline{A} &= \{x\mid x\notin A\} && \text{(complement)}.
\end{align*}

\begin{example}
Let $A=\{1,2,3,4\}$ and $B=\{3,4,5\}$.  
Then
$A\cap B=\{3,4\}$,
$A\cup B=\{1,2,3,4,5\}$,
$A\setminus B=\{1,2\}$,
$\overline{A}$ is the set of elements of the universe not in $A$.
\end{example}

\subsection*{Subset and Disjointness}
$A\subseteq B$ means every element of $A$ also lies in $B$.
Two sets are \emph{disjoint} if $A\cap B=\emptyset$.

\begin{corollary}
If $A\subseteq B$ and $B\subseteq C$, then $A\subseteq C$.
\end{corollary}

\begin{proof}
Take $x\in A$.  
Since $A\subseteq B$, $x\in B$.
Because $B\subseteq C$, we also have $x\in C$.
\end{proof}

\subsection*{Symmetric Difference}
The \emph{symmetric difference} of $A$ and $B$ is
\[
A\triangle B = (A\setminus B)\cup (B\setminus A),
\]
the set of elements belonging to exactly one of $A$ or $B$.

\begin{example}
If $A=\{1,2,3\}$ and $B=\{2,3,4\}$,
then $A\triangle B=\{1,4\}$.
\end{example}

\subsection*{All Set Laws}
\begin{align*}
&\textbf{Identity:}& &A \cap U = A,\quad A \cup \emptyset = A.\\
&\textbf{Domination:}& &A \cup U = U,\quad A \cap \emptyset = \emptyset.\\
&\textbf{Idempotent:}& &A \cap A = A,\quad A \cup A = A.\\
&\textbf{Complementation:}& &\overline{\overline{A}} = A.\\
&\textbf{Commutative:}& &A \cap B = B \cap A,\quad A \cup B = B \cup A.\\
&\textbf{Associative:}& &A \cap (B \cap C) = (A \cap B) \cap C,\\
&&&A \cup (B \cup C) = (A \cup B) \cup C.\\
&\textbf{Distributive:}& &A \cap (B \cup C) = (A \cap B) \cup (A \cap C),\\
&&&A \cup (B \cap C) = (A \cup B) \cap (A \cup C).\\
&\textbf{De Morgan:}& &\overline{A \cap B} = \overline{A} \cup \overline{B},\quad
\overline{A \cup B} = \overline{A} \cap \overline{B}.\\
&\textbf{Absorption:}& &A \cup (A \cap B) = A,\quad A \cap (A \cup B) = A.\\
&\textbf{Complement Laws:}& &A \cup \overline{A} = U,\quad A \cap \overline{A} = \emptyset.
\end{align*}

\begin{review}
\begin{itemize}
  \item Predicates let us describe infinite families of propositions with one formula.
  \item Sets collect objects satisfying a property, written $\{x\mid \varphi(x)\}$.
  \item Bound variables can be renamed freely; free variables cannot.
  \item Truth sets turn logical operations into unions, intersections, and complements.
  \item Key results include subset transitivity and De Morgan’s laws.
\end{itemize}
\end{review}

\begin{example}
   Claim: $A \textbackslash (A \cap B) = A \textbackslash B$
   \[
       = \{x | x \in A \land x \not\in A \land B\}
   .\] Definition of $\textbackslash$.
    \[
        = \{x | x \in A \land \neg(x \in A \cap B)\}
   .\] Definition of $\not\in $.
   \[
       = \{x | x \in A \land \neg(x \in A \land x \in B)\}
   .\] Definition of $\cap$.
    \[
        = \{x | x \in  A \land (x \not\in A \lor x \not\in B)\}
   .\] De Morgan's.
   \[
       = \{ x | (x \in A \land x \not\in A) \lor (x \in  A \land x \not\in B) \}
   .\] Distributive.
   \[
       = \{x | x \in A \land x \not\in B\}
   .\] Contradiction.
    \[
       = A \textbackslash B 
    .\] 
\end{example}

\section{Quantifiers}
\label{sec:Quantifiers}

Suppose $P(x)$ stands for “$x$ is prime.” To understand quantified statements, we ask:
\begin{itemize}
    \item Is $P(x)$ true for every $x$?
    \item Is $P(x)$ true for some $x$?
    \item For which $x$ does it make sense to ask whether $P(x)$ is true?
\end{itemize}

\begin{definition}[Universe of Discourse]
The \emph{universe of discourse} is the set of all possible values that a variable can take.
All statements involving that variable are considered relative to this universe.
\end{definition}

\subsection*{Universal Quantifier}
The \emph{universal quantifier} is written as
\[
\forall x, P(x),
\]
and is read as “for all $x$, $P(x)$ holds.”

\begin{itemize}
    \item It is true if $P(x)$ is true for every element in the universe of discourse.
    \item It is false if there exists at least one element in the universe for which $P(x)$ is false.
\end{itemize}

\begin{example}
Let the universe $U = \{2, 3, 4, 5\}$. Then
\[
\forall x \in U, \; P(x) \equiv P(2)\land P(3)\land P(4)\land P(5).
\]
Here, $P(4)$ is false, so the universally quantified statement is false.
\end{example}

\subsection*{Existential Quantifier}
The \emph{existential quantifier} is written as
\[
\exists x, P(x),
\]
and is read as “there exists an $x$ such that $P(x)$ holds.”

\begin{itemize}
    \item It is true if there is at least one element in the universe for which $P(x)$ is true.
    \item It is false only if $P(x)$ is false for every element in the universe.
\end{itemize}

\begin{example}
Using the same universe $U = \{2, 3, 4, 5\}$:
\[
\exists x \in U, \; P(x) \equiv P(2) \vee P(3) \vee P(4) \vee P(5),
\]
which is true because $P(2)$, $P(3)$, and $P(5)$ are true.
\end{example}

\subsection*{Negating Quantifiers}
Quantifiers interact with negation in a very structured way:
\[
\neg (\forall x, P(x)) \equiv \exists x, \neg P(x), \qquad
\neg (\exists x, P(x)) \equiv \forall x, \neg P(x).
\]
\begin{example}
If $P(x)$ = “$x$ is even” and the universe is $\{1,2,3\}$, then
\[
\neg (\forall x, P(x)) \equiv \exists x, \neg P(x),
\]
is true because $1$ is not even.
\end{example}

\subsection*{Finite Universe Expansion}
For a finite universe $U = \{a,b,c,d\}$, quantified statements can be expanded explicitly:
\[
\forall x \in U, P(x) \equiv P(a) \wedge P(b) \wedge P(c) \wedge P(d),
\]
\[
\exists x \in U, P(x) \equiv P(a) \vee P(b) \vee P(c) \vee P(d).
\]

\subsection*{Order of Quantifiers}
Quantifiers can be combined, and their order matters.
For example:
\[
\forall x \exists y, R(x,y) \quad \text{vs} \quad \exists y \forall x, R(x,y)
\]
may have very different meanings. One asserts that \emph{for each $x$ we can find a $y$}, while the other asserts that \emph{there is a single $y$ that works for all $x$}.

\begin{example}
Let $R(x,y)$ = “$x + y = 5$” and universe $U = \{1,2,3\}$. Then
\[
\forall x \exists y, x+y=5 \quad \text{is true (choose }y=5-x\text{ for each }x\text{),}
\]
\[
\exists y \forall x, x+y=5 \quad \text{is false (no single $y$ works for all $x$).}
\end{example}

\begin{review}[Logical Connectives and Quantifiers]
\[
\begin{array}{ccl}
\neg P & : & \text{“not $P$” (negation, highest precedence)} \\
P \wedge Q & : & \text{“$P$ and $Q$” (conjunction)} \\
P \vee Q & : & \text{“$P$ or $Q$” (inclusive disjunction)} \\
P \rightarrow Q & : & \text{“if $P$, then $Q$” (implication)} \\
P \leftrightarrow Q & : & \text{“$P$ if and only if $Q$” (biconditional)} \\
\forall, \exists & : & \text{“for all, there exists” (quantifiers, lowest precedence)}
\end{array}
\]
\end{review}

\begin{exercise}
\begin{itemize}
    \item $\exists x, M(x) \land B(x)$ with a universe set of all people, and
        $M(x)$ stands for "$x$ is a man", $B(x)$ stands for "$x$ has brown
        hair." 
    \item $\forall x, M(x) \to B(x)$, with universe, $M$, and $B$ as seen above.
        \item $\forall x, M(x) \land B(x)$ with universe, $M$, and $B$ as seen
            above.
            \item $\forall x, L(x,y)$ with a universe set of all people, and
                $L(x,y)$ stands for "$x$ likes $y$".
\end{itemize}

\end{exercise}
