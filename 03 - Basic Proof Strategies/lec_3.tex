\section{Induction: From Patterns to Proofs}
\label{sec:induction}

Induction is the bridge from testing a few cases to proving \emph{all} cases.
It replaces an infinite chore with two finite checks: a base case and an
inductive step. Once those two links are solid, the whole chain holds.

\begin{review}[A first puzzle]
One million balls are labeled \(B_1, B_2, \ldots, B_{1{,}000{,}000}\). Suppose
\begin{enumerate}[nosep]
    \item If \(B_i\) is red for any \(1 \leq i \leq 999{,}999\), then \(B_{i+1}\)
    is red.
    \item Ball \(B_1\) is red.
\end{enumerate}
The rules force a cascade: \(B_1\) is red, so \(B_2\) is red, so \(B_3\) is red,
and so on. Eventually \(B_{1{,}000{,}000}\) is red too. This is induction in
street clothes.
\end{review}

\subsection{Simple (weak) induction}
\begin{definition}[Simple induction]
Let \(P(n)\) be a statement about an integer \(n\). To establish
\(\forall n \geq b\, P(n)\):
\begin{enumerate}
    \item \textbf{Base case:} Prove \(P(b)\).
    \item \textbf{Inductive step:} Prove \(\forall k \geq b,\, P(k) \to P(k+1)\).
    In the step you may assume \(P(k)\) (the \emph{induction hypothesis}) and
    must deduce \(P(k+1)\).
\end{enumerate}
If both succeed, every \(P(n)\) with \(n \geq b\) is true.
\end{definition}

\begin{remark}[Bounded quantifiers]
The shorthand \(\forall n \geq b, P(n)\) means \(\forall n,\, n \geq b \to P(n)\).
Similarly, \(\exists n \geq b, P(n)\) abbreviates \(\exists n,\, n \geq b \land
P(n)\).
\end{remark}

\subsection{Fast intuition with examples}
Each example follows the same rhythm: name \(P(n)\), check the base case, then
transform the \(k\)-case into the \(k+1\)-case.

\begin{example}[Sum of the first odd numbers]
Claim: for all \(n \geq 1\),
\(1 + 3 + \cdots + (2n-1) = n^2\).
\begin{proof}
Let \(P(n)\) denote the claim. The base case \(P(1)\) is \(1 = 1^2\).
Assume \(P(k)\): the partial sum is \(k^2\). Adding the next odd number,
\begin{align*}
    1 + 3 + \cdots + (2k-1) + (2(k+1)-1) &= k^2 + 2(k+1) - 1 \\
    &= (k+1)^2.
\end{align*}
Thus \(P(k) \to P(k+1)\), and the formula holds for every \(n\geq 1\).
\end{proof}
\end{example}

\begin{example}[Triangular numbers]
For every \(n \geq 1\), \(1 + 2 + 3 + \cdots + n = \tfrac{n(n+1)}{2}\).
\begin{proof}
The base case \(n=1\) is immediate. Assume the sum up to \(k\) equals
\(k(k+1)/2\). Then
\[
    1 + 2 + \cdots + k + (k+1)
    = \frac{k(k+1)}{2} + (k+1)
    = \frac{(k+1)(k+2)}{2},
\]
which is \(P(k+1)\).
\end{proof}
\end{example}

\begin{example}[Finite geometric series]
For every \(n \geq 1\), \(1 + 2 + 2^2 + \cdots + 2^n = 2^{n+1} - 1\).
\begin{proof}
The base case \(n=1\) reads \(1+2=2^2-1\). Assuming the sum through \(2^k\)
equals \(2^{k+1}-1\), adding \(2^{k+1}\) yields
\[
    (2^{k+1}-1) + 2^{k+1} = 2^{k+2}-1 = 2^{(k+1)+1}-1.
\]
\end{proof}
\end{example}

\begin{example}[A basic inequality]
For every \(n \geq 1\), \(n < 2^n\).
\begin{proof}
The claim is clear at \(n=1\). Assume \(k < 2^k\). Then
\[
    k+1 \leq 2k < 2 \cdot 2^k = 2^{k+1},
\]
so the inequality holds for \(k+1\) as well.
\end{proof}
\end{example}

\begin{remark}[Choosing a later base case]
Sometimes the statement first becomes true at \(b>1\). The same induction
pattern applies; simply start at \(b\).
\end{remark}

\begin{example}[Exponential beats linear]
For \(n \geq 3\), \(2n < 2^n\).
\begin{proof}
Verify \(2\cdot3 < 2^3\). Assume \(2k < 2^k\) with \(k \geq 3\). Then
\[
    2(k+1) = 2k + 2 \leq 2k + 2k = 2 \cdot 2k < 2^{k+1}.
\]
\end{proof}
\end{example}

\begin{example}[Exponential beats quadratic]
For \(n \geq 5\), \(n^2 < 2^n\).
\begin{proof}
Check the base case at \(n=5\). Assume \(k^2 < 2^k\) for \(k \geq 5\).
Using \(2k < 2^k\) (valid for \(k \geq 3\)),
\begin{align*}
    (k+1)^2 &= k^2 + 2k + 1 \\
            &< 2^k + 2^k + 1 \leq 2^k + 2^k + 2^k = 2^{k+1}.
\end{align*}
\end{proof}
\end{example}

\subsection{When one step is not enough}
Some proofs benefit from knowing \emph{all} smaller cases, not just the previous
one. The classic coin problem illustrates this.

\begin{example}[Building every amount from \$12]
Claim: every integer \(n \geq 12\) can be expressed as \(3x + 7y\) with
non-negative integers \(x,y\).
\begin{proof}[Idea]
A direct step from \(P(k)\) to \(P(k+1)\) works but requires fiddly casework.
Strong induction yields a cleaner proof; see Example~\ref{ex:coin-strong}.
\end{proof}
\end{example}

\subsection{Strong induction}
\begin{definition}[Strong induction]
To prove \(\forall n \geq b,\, P(n)\), it suffices to show
\[
    \forall k \geq b,\; \Bigl(\forall i \text{ with } b \leq i < k,\, P(i)\Bigr)
    \to P(k).
\]
Here the induction hypothesis grants \emph{all} previous cases.
\end{definition}

\begin{example}[Coin problem revisited]\label{ex:coin-strong}
Let \(P(n)\) be ``\(n = 3x + 7y\) for some non-negative integers \(x,y\).''
\begin{proof}
For \(12 \leq n \leq 14\), explicit combinations verify \(P(n)\):
\(12 = 4 \times 3\), \(13 = 7 + 2 \times 3\), and \(14 = 2 \times 7\).
Assume now that \(P(i)\) holds for all \(12 \leq i < k\) with \(k \geq 15\).
Because \(k-3 \geq 12\), the hypothesis provides \(k-3 = 3x + 7y\). Then
\(k = 3(x+1) + 7y\), so \(P(k)\) follows. Thus every \(n \geq 12\) is
representable.
\end{proof}
\end{example}

\begin{example}[Prime factorisation]
For every integer \(n \geq 2\), \(n\) is prime or a product of at least two
primes.
\begin{proof}
Assume the claim for all \(2 \leq i < k\). If \(k\) is prime, we are done. If
not, write \(k = ab\) with \(1 < a,b < k\). By the hypothesis, \(a\) and \(b\)
are prime or products of primes, so their product \(k\) is a product of at least
two primes.
\end{proof}
\end{example}

\begin{example}[Binary (base-2) representation]
Every positive integer is a sum of \emph{distinct} powers of two.
\begin{proof}
Assume all integers below \(k\) admit such a representation. If \(k\) is even,
write \(k = 2a\) with \(1 \leq a < k\); by hypothesis,
\(a = 2^{e_1} + \cdots + 2^{e_m}\) with distinct exponents, so
\(k = 2^{e_1+1} + \cdots + 2^{e_m+1}\). If \(k\) is odd and greater than one,
write \(k = 2b + 1\) with \(1 \leq b < k\) and append a \(2^0\) term to the
representation of \(b\). The case \(k=1\) is \(1 = 2^0\). Distinctness of
exponents is preserved in all cases.
\end{proof}
\end{example}

\begin{remark}[What to remember]
\begin{itemize}
    \item Induction formalises ``check the first case, then show the machine can
    keep running.''
    \item Match the base case to the first value where the statement is claimed.
    \item Simple induction needs one prior case; strong induction may use all
    prior cases to simplify the step.
    \item Inequalities often combine induction with previously established
    bounds.
    \item Decomposition problems (coins, prime factors, binary expansions) are
    natural strong-induction territory.
\end{itemize}
\end{remark}

\clearpage
