\chapter{Basic Proof Strategies}
\label{cha:basic-proof-strategies}

In mathematics, there are several common techniques used to establish the truth of statements.
Each technique corresponds to a logical structure or way of reasoning about implications,
negations, and cases. In this chapter, we review the most fundamental proof strategies:
direct proof, proof by contrapositive, proof by contradiction, and proof by cases.

\section{Direct Proof}
\label{sec:direct-proof}

\begin{quote}
\textbf{Idea:} Assume \(P\) is true, then use logical reasoning to show that \(Q\) must also be true.
\end{quote}

A \emph{direct proof} is the most straightforward type of proof for a conditional statement
of the form \(P \to Q\). Recall that such an implication is false only when \(P\) is true
and \(Q\) is false. Therefore, to show that \(P \to Q\) is always true, we assume \(P\) holds
and logically deduce that \(Q\) must hold as well.

This technique often appears in universally quantified statements such as:
\[
\forall x \in D, \; P(x) \to Q(x).
\]
To prove this directly, follow these general steps:
\begin{enumerate}
    \item Let \(x\) be an \emph{arbitrary element} of the domain \(D\).
    \item Assume \(P(x)\) is true.
    \item Use logical reasoning, definitions, or known facts to show that \(Q(x)\) must be true.
\end{enumerate}
If this reasoning works for an arbitrary \(x \in D\), the implication holds for all \(x\).

\begin{example}[Direct Proof]
\textbf{Proposition.} For any person, if they are born in Canada, then they are born in North America.

\begin{proof}
Let \(D\) be the set of all people.
Define:
\[
\begin{aligned}
P(x) &: \text{``Person $x$ was born in Canada.''}\\
Q(x) &: \text{``Person $x$ was born in North America.''}
\end{aligned}
\]

We wish to prove:
\[
\forall x \in D, \; P(x) \to Q(x).
\]

Let \(x\) be an arbitrary person in \(D\).
Assume \(P(x)\) is true; that is, person \(x\) was born in Canada.
By geographical fact, Canada is a country located within the continent of North America.
Therefore, person \(x\) was also born in North America — meaning \(Q(x)\) holds.

Since \(x\) was arbitrary, we conclude that \(P(x) \to Q(x)\) is true for all \(x \in D\).
\end{proof}
\end{example}

\section{Proof by Contrapositive}
\label{sec:proof-by-contrapositive}

\begin{quote}
\textbf{Idea:} Instead of proving \(P \to Q\) directly, prove its contrapositive \(\neg Q \to \neg P\).
\end{quote}

A statement of the form \(P \to Q\) is \emph{logically equivalent} to its contrapositive:
\[
P \to Q \quad \equiv \quad \neg Q \to \neg P.
\]
Thus, proving the contrapositive automatically proves the original statement.

This approach is often useful when the negation of the conclusion (\(\neg Q\)) is easier to handle
than the hypothesis (\(P\)) itself.

\subsection*{Steps for Proving by Contrapositive}
To prove \(\forall x \in D, \; P(x) \to Q(x)\) by contrapositive:
\begin{enumerate}
    \item Let \(x\) be an arbitrary element of the domain \(D\).
    \item Assume \(\neg Q(x)\) is true.
    \item Use logic and definitions to show that \(\neg P(x)\) must follow.
    \item Conclude that \(\neg Q(x) \to \neg P(x)\) holds for all \(x\).
    \item By logical equivalence, the original implication \(P(x) \to Q(x)\) is true.
\end{enumerate}

\begin{example}[Proof by Contrapositive]
\textbf{Proposition.} If a person is not born in North America, then they are not born in Canada.

\begin{proof}
Let the domain \(D\) be the set of all people. Define:
\[
\begin{aligned}
P(x) &: \text{``Person $x$ was born in Canada.''}\\
Q(x) &: \text{``Person $x$ was born in North America.''}
\end{aligned}
\]
We wish to prove \(P(x) \to Q(x)\) by proving its contrapositive \(\neg Q(x) \to \neg P(x)\).

Let \(x \in D\) be arbitrary and assume \(\neg Q(x)\) is true; that is, person \(x\) was \emph{not}
born in North America. Since Canada is geographically located in North America, if \(x\) were born
in Canada, then \(x\) would necessarily have been born in North America. This contradicts our assumption.
Therefore, \(x\) cannot have been born in Canada, meaning \(\neg P(x)\) holds.

Hence, \(\neg Q(x) \to \neg P(x)\) is true for all \(x\), and thus \(P(x) \to Q(x)\) follows by
logical equivalence.
\end{proof}
\end{example}

\section{Proof by Contradiction}
\label{sec:proof-by-contradiction}

\begin{quote}
\textbf{Idea:} Assume the negation of what you want to prove, and show that this assumption leads to a contradiction.
\end{quote}

A proof by contradiction works by assuming the opposite of the desired conclusion.
If this assumption logically leads to a contradiction, then the assumption must be false,
and the desired statement must therefore be true.

Formally, to prove \(P\), assume \(\neg P\) and derive a contradiction (such as \(R \wedge \neg R\)).
Once a contradiction appears, we conclude that \(\neg P\) is false and thus \(P\) is true.

\begin{example}[Proof by Contradiction]
\textbf{Proposition.} There is no smallest positive real number.

\begin{proof}
Suppose, for the sake of contradiction, that there \emph{is} a smallest positive real number.
Let this number be \(s > 0\), so that for every positive real number \(x\), we have \(x \ge s\).

Now consider the number \(\dfrac{s}{2}\). Since \(s > 0\), it follows that \(\dfrac{s}{2} > 0\).
However, \(\dfrac{s}{2} < s\), which contradicts the assumption that \(s\) is the smallest
positive real number.

Hence, our assumption is false. Therefore, there is \emph{no smallest positive real number}.
\end{proof}
\end{example}

\section{Proof by Cases (Exhaustion)}
\label{sec:proof-by-cases}

\begin{quote}
\textbf{Idea:} If a statement depends on multiple possibilities, prove each case separately.
\end{quote}

Suppose we want to prove:
\[
(X \lor Y) \to Q.
\]
We can show this by establishing both
\[
X \to Q \quad \text{and} \quad Y \to Q.
\]
If \(Q\) follows in all possible cases, then \(Q\) holds whenever \(X \lor Y\) does.

\begin{example}[Proof by Cases]
\textbf{Proposition.} If a number is divisible by 4 or divisible by 6, then it is divisible by 2.

\begin{proof}
Let \(n\) be an arbitrary integer.

\textbf{Case 1:} Suppose \(n\) is divisible by 4.\\
Then \(n = 4k\) for some integer \(k\). Since \(4 = 2 \times 2\), we can write
\(n = 2(2k)\), which shows \(n\) is divisible by 2.

\textbf{Case 2:} Suppose \(n\) is divisible by 6.\\
Then \(n = 6m\) for some integer \(m\). Since \(6 = 2 \times 3\), we can write
\(n = 2(3m)\), so \(n\) is also divisible by 2.

In both cases, \(n\) is divisible by 2. Therefore, if \(n\) is divisible by 4 or 6,
then \(n\) is divisible by 2.
\end{proof}
\end{example}

\section{Key Definitions}
\label{sec:key-definitions}

\begin{definition}[Theorem]
A statement that has been proved true based on axioms, definitions, and previously established results.
\end{definition}

\begin{definition}[Axiom]
A self-evident or universally accepted statement taken to be true without proof.
\end{definition}

\begin{definition}[Identity]
An equation that holds true for all values of the variable(s) involved, such as \(\sin^2 x + \cos^2 x = 1\).
\end{definition}

\begin{definition}[Proof]
A logical and systematic argument that demonstrates the truth of a mathematical statement.
\end{definition}

\begin{definition}[Tautology]
A propositional formula that is always true, regardless of the truth values of its components.
For example: \(A \lor \neg A\).
\end{definition}

\begin{definition}[Rational Number]
A number that can be expressed as a fraction \(\dfrac{p}{q}\) where \(p, q \in \mathbb{Z}\)
and \(q \ne 0\), and \(p\) and \(q\) are relatively prime.
\end{definition}
