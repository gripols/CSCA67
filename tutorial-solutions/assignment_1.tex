\documentclass{article}

\usepackage{fancyhdr}
\usepackage{extramarks}
\usepackage{amsmath}
\usepackage{amsthm}
\usepackage{amsfonts}
\usepackage{amssymb}
\usepackage{tcolorbox}
\usepackage{enumitem} 
\usepackage{tikz}
\usepackage[plain]{algorithm}
\usepackage{algpseudocode}

\usetikzlibrary{automata,positioning}

%
% Basic Document Settings
%

\topmargin=-0.45in
\evensidemargin=0in
\oddsidemargin=0in
\textwidth=6.5in
\textheight=9.0in
\headsep=0.25in

\linespread{1.1}

\pagestyle{fancy}
\lhead{\hmwkAuthorName}
\chead{\hmwkClass\ (\hmwkClassInstructor\ \hmwkClassTime): \hmwkTitle}
\rhead{\firstxmark}
\lfoot{\lastxmark}
\cfoot{\thepage}

\renewcommand\headrulewidth{0.4pt}
\renewcommand\footrulewidth{0.4pt}

\setlength\parindent{0pt}

%
% Create Problem Sections
%

\newcommand{\enterProblemHeader}[1]{
    \nobreak\extramarks{}{Problem \arabic{#1} continued on next page\ldots}\nobreak{}
    \nobreak\extramarks{Problem \arabic{#1} (continued)}{Problem \arabic{#1} continued on next page\ldots}\nobreak{}
}

\newcommand{\exitProblemHeader}[1]{
    \nobreak\extramarks{Problem \arabic{#1} (continued)}{Problem \arabic{#1} continued on next page\ldots}\nobreak{}
    \stepcounter{#1}
    \nobreak\extramarks{Problem \arabic{#1}}{}\nobreak{}
}

\setcounter{secnumdepth}{0}
\newcounter{partCounter}
\newcounter{homeworkProblemCounter}
\setcounter{homeworkProblemCounter}{1}
\nobreak\extramarks{Problem \arabic{homeworkProblemCounter}}{}\nobreak{}

%
% Homework Problem Environment
%
% This environment takes an optional argument. When given, it will adjust the
% problem counter. This is useful for when the problems given for your
% assignment aren't sequential. See the last 3 problems of this template for an
% example.
%
\newenvironment{homeworkProblem}[1][-1]{
    \ifnum#1>0
        \setcounter{homeworkProblemCounter}{#1}
    \fi
    \section{Problem \arabic{homeworkProblemCounter}}
    \setcounter{partCounter}{1}
    \enterProblemHeader{homeworkProblemCounter}
}{
    \exitProblemHeader{homeworkProblemCounter}
}

% Solution box environment
\newenvironment{solutionbox}[1][4cm]{%
  \begin{tcolorbox}[colback=white, colframe=black!75, title=Solution,
  enhanced, sharp corners, height=#1, breakable]
}{\end{tcolorbox}}

%
% Homework Details
%   - Title
%   - Due date
%   - Class
%   - Section/Time
%   - Instructor
%   - Author
%

\newcommand{\hmwkTitle}{Assignment\ \#1}
\newcommand{\hmwkDueDate}{September 28, 2025}
\newcommand{\hmwkClass}{CSCA67}
\newcommand{\hmwkClassTime}{Ch. 1}
\newcommand{\hmwkClassInstructor}{Professor Anya Tafliovich}
\newcommand{\hmwkAuthorName}{\textbf{Gosha Polstvin}}

%
% Title Page
%

\title{
    \vspace{2in}
    \textmd{\textbf{\hmwkClass:\ \hmwkTitle}}\\
    \normalsize\vspace{0.1in}\small{Due\ on\ \hmwkDueDate\ at 11:59pm}\\
    \vspace{0.1in}\large{\textit{\hmwkClassInstructor\ \hmwkClassTime}}
    \vspace{3in}
}

\author{\hmwkAuthorName}
\date{}

\renewcommand{\part}[1]{\textbf{\large Part \Alph{partCounter}}\stepcounter{partCounter}\\}
\newcommand{\impl}{\rightarrow}
\newcommand{\liff}{\leftrightarrow}
\newcommand{\T}{\text{T}}
\newcommand{\F}{\text{F}}
\newcommand{\entails}{\models}
\newcommand{\nentails}{\nvDash}
%
% Various Helper Commands
%

% Useful for algorithms
\newcommand{\alg}[1]{\textsc{\bfseries \footnotesize #1}}

% For derivatives
\newcommand{\deriv}[1]{\frac{\mathrm{d}}{\mathrm{d}x} (#1)}

% For partial derivatives
\newcommand{\pderiv}[2]{\frac{\partial}{\partial #1} (#2)}

% Integral dx
\newcommand{\dx}{\mathrm{d}x}

% Alias for the Solution section header
\newcommand{\solution}{\textbf{\large Solution}}

% Probability commands: Expectation, Variance, Covariance, Bias
\newcommand{\E}{\mathrm{E}}
\newcommand{\Var}{\mathrm{Var}}
\newcommand{\Cov}{\mathrm{Cov}}
\newcommand{\Bias}{\mathrm{Bias}}

\begin{document}

\maketitle

\pagebreak

% ------------------- QUESTION 1 -------------------
\begin{document}
\maketitle

\section*{Algebraic proofs for Question 1}

\subsection*{1. $(a\to b)\land(b\to c)$ vs. $a\to c$}

\[
\begin{aligned}
(a\to b)\land(b\to c) &\equiv (\neg a \lor b)\land(\neg b \lor c) \\
&\Rightarrow \neg a \lor c \equiv a\to c.
\end{aligned}
\]

\textbf{Failure of converse.} Take $(a,b,c)=(\F,\T,\F)$. Then $a\to c=\T$, but $(a\to b)\land(b\to c)=\F$, so $a\to c$ does not imply $(a\to b)\land(b\to c)$.

\vspace{1ex}
\subsection*{2. $a\land(a\to b)$ vs. $a\to b$}

\[
a\land(a\to b) \Rightarrow a\to b,
\]
since $a\to b$ is a conjunct of the left-hand formula.

\textbf{Failure of converse.} Take $(a,b)=(\F,\F)$. Then $a\to b=\T$ but $a\land(a\to b)=\F$, so $a\to b$ does not imply $a\land(a\to b)$.

\vspace{1ex}
\subsection*{3. $(a\to b)\land(a\to c)$ vs. $a\to(b\land c)$}

\[
\begin{aligned}
(a\to b)\land(a\to c) &\equiv (\neg a \lor b)\land(\neg a \lor c) \\
&\equiv \neg a \lor (b\land c) \equiv a\to(b\land c),
\end{aligned}
\]
so the formulas are equivalent.

\vspace{1ex}
\subsection*{4. $(a\to c)\land(b\to c)$ vs. $(a\lor b)\to c$}

\[
\begin{aligned}
(a\to c)\land(b\to c) &\equiv (\neg a \lor c)\land(\neg b \lor c) \\
&\equiv (\neg a \land \neg b) \lor c \equiv \neg(a\lor b) \lor c \equiv (a\lor b)\to c,
\end{aligned}
\]
so they are equivalent.

\vspace{1ex}
\subsection*{5. $a\liff b$ vs. $(a\land b)\lor(\neg a\land\neg b)$}

By definition of biconditional:
\[
a\liff b \equiv (a\land b)\lor(\neg a \land \neg b),
\]
so the formulas are identical.

\vspace{1ex}
\subsection*{6. $a\to(b\to(c\to d))$ vs. $((a\land b)\land c)\to d$}

\[
\begin{aligned}
a\to(b\to(c\to d)) &\equiv \neg a \lor (\neg b \lor (\neg c \lor d)) \\
&\equiv \neg(a\land b\land c) \lor d \equiv ((a\land b)\land c)\to d,
\end{aligned}
\]
so they are equivalent.

\vspace{1ex}
\subsection*{7. $(a\to b)\lor(b\to a)$ vs. $a\liff b$}

\textbf{Left-hand side is a tautology:}
\[
\begin{aligned}
(a\to b)\lor(b\to a) &\equiv (\neg a \lor b) \lor (\neg b \lor a) \\
&\equiv (\neg a \lor a) \lor (\neg b \lor b) \equiv \top.
\end{aligned}
\]

\textbf{Right-hand side is not a tautology.} Example: $(a,b)=(\T,\F)$ gives $a\liff b=\F$. Hence they are not equivalent.

\vspace{1ex}
\subsection*{8. $a\liff b$ vs. $\neg a\liff \neg b$}

\[
\begin{aligned}
\neg a\liff \neg b &\equiv (\neg a \land \neg b) \lor (a\land b) \\
&\equiv (a\land b)\lor(\neg a\land\neg b) \equiv a\liff b,
\end{aligned}
\]
so they are equivalent.

\vspace{1ex}
\subsection*{9. $(a\land b)\to(c\land d)$ vs. $((a\to c)\land(a\to d))\land((b\to c)\land(b\to d))$}

\textbf{RHS implies LHS:}
\[
\begin{aligned}
\text{RHS} &\equiv (\neg a\lor c)\land(\neg a\lor d)\land(\neg b\lor c)\land(\neg b\lor d) \\
&\equiv ((\neg a\lor c)\land(\neg b\lor c)) \land ((\neg a\lor d)\land(\neg b\lor d)) \\
&\equiv ((\neg a \land \neg b)\lor c) \land ((\neg a\land \neg b)\lor d) \\
&\equiv (\neg a\land \neg b)\lor(c\land d) \equiv (a\land b)\to(c\land d).
\end{aligned}
\]

\textbf{Failure of converse.} Example: $(a,b,c,d)=(\T,\F,\F,\F)$ gives LHS true, RHS false.

\vspace{1ex}
\subsection*{10. $(a\lor b)\to(c\land d)$ vs. $((a\to c)\land(a\to d))\land((b\to c)\land(b\to d))$}

\[
\begin{aligned}
(a\lor b)\to(c\land d) &\equiv (\neg a \land \neg b) \lor (c\land d), \\
\text{RHS} &\equiv (\neg a \lor c)\land(\neg a \lor d)\land(\neg b \lor c)\land(\neg b \lor d) \\
&\equiv ((\neg a\lor c)\land(\neg b\lor c)) \land ((\neg a\lor d)\land(\neg b\lor d)) \\
&\equiv ((\neg a \land \neg b)\lor c)\land((\neg a\land \neg b)\lor d) \\
&\equiv (\neg a \land \neg b)\lor(c\land d) \equiv (a\lor b)\to(c\land d),
\end{aligned}
\]
so they are equivalent.

\bigskip

\section*{Question 2. Tautology / contradiction / neither}
We classify each formula and give succinct rigorous arguments.

\subsection*{1. $(a\to b)\lor(b\to a)$.}
\textbf{Classification:} \emph{Tautology.}

\begin{proof}
As in Question 1.7,
\[
(\neg a\lor b)\lor(\neg b\lor a)\equiv(\neg a\lor a)\lor(\neg b\lor b)\equiv\top,
\]
so the formula is true under every valuation.
\end{proof}

\vspace{1ex}

\subsection*{2. $(((a\to b)\land(b\to c))\land a)\land\neg c$.}
\textbf{Classification:} \emph{Contradiction} (unsatisfiable).

\begin{proof}
Assume the whole conjunction is true. Then from $(a\to b)\land(b\to c)$ and $a$ we can derive $b$ and then $c$ by successive modus ponens. But the formula also contains $\neg c$ as a conjunct, hence we deduce both $c$ and $\neg c$, contradiction. Therefore there is no valuation that makes the formula true; it is a contradiction.
\end{proof}

\vspace{1ex}

\subsection*{3. $(((a\to b)\land(b\to c))\land a)\land c$.}
\textbf{Classification:} \emph{Neither} (satisfiable but not a tautology).

\begin{proof}
Satisfiable: take $a=b=c=\T$. Then each of $a\to b$, $b\to c$ is $\T$, and $a\land c$ is $\T$, so the whole conjunction is $\T$. Not a tautology: take $a=\F,b=\F,c=\F$. Then the conjunct $a$ is $\F$, so the whole conjunction is $\F$. Hence neither a tautology nor a contradiction.
\end{proof}

\vspace{1ex}

\subsection*{4. $a\to\neg a$.}
\textbf{Classification:} \emph{Neither}.

\begin{proof}
If $a=\F$ the implication $a\to\neg a$ is $\T$ (antecedent false). If $a=\T$ the implication is $\T\to\F=\F$. Thus the formula is satisfiable but not always true; hence neither.
\end{proof}

\vspace{1ex}

\subsection*{5. $(a\land(a\to b))\to b$.}
\textbf{Classification:} \emph{Tautology}.

\begin{proof}
Suppose $a\land(a\to b)$ holds. Then $a$ holds and $a\to b$ holds, so by modus ponens $b$ holds. Therefore the implication from $a\land(a\to b)$ to $b$ is always true, i.e.\ a tautology.
Algebraically one can expand the antecedent as $a\land(\neg a\lor b)$ and simplify to see the implication is always true, but the straightforward rule-based derivation via modus ponens is immediate and rigorous.
\end{proof}

\bigskip

\section*{Question 3. Deductive reasoning (validity / proofs)}

For each argument we indicate validity. If valid, we present (i) a truth-table style check covering all rows where the premises are all true, and (ii) a rules-of-inference (natural deduction) proof. If invalid, we give an explicit counterexample valuation.

%======================
% (A)
%======================
\subsection*{(A) Truth Table}

\begin{enumerate} \item If the weather is good, I either go running or swimming. \quad $W\to(R\lor S)$. \item I don’t go running and swimming at the same time. \quad $\neg(R\land S)$. \end{enumerate}
\textbf{Validity:} \emph{Valid.}

\begin{enumerate}
  \item $W\to(R\lor S)$ \hfill Premise
  \item $\neg(R\land S)$ \hfill Premise
  \item From $\neg(R\land S)$ we get $R\to\neg S$ and $S\to\neg R$. \hfill (Equivalently: $\neg(R\land S)\equiv \neg R\lor\neg S$, from which each implication follows; or prove each implication by conditional proof.)
  \item To prove $(W\land R)\to\neg S$: assume $W\land R$.
  \begin{enumerate}
    \item From $W\land R$ infer $W$ and $R$.
    \item From $W$ and $W\to(R\lor S)$ infer $R\lor S$.
    \item From $R$ and $R\to\neg S$ infer $\neg S$.
  \end{enumerate}
  Discharging the assumption $W\land R$ yields $(W\land R)\to\neg S$.
  \item The proof of $(W\land S)\to\neg R$ is symmetric.
\end{enumerate}
\[
\begin{tabular}{ccc c c c c c}
\toprule

R & S & W & $W\to(R\lor S)$ & $\neg(R\land S)$ & \text{All Premises?} & $(W\land R)\to\neg S$ & $(W\land S)\to\neg R$ \\
\midrule
F & F & F & T & T & T & T & T \\
F & F & T & F & T & F & T & T \\
F & T & F & T & T & T & T & T \\
F & T & T & T & T & T & T & T \\
T & F & F & T & T & T & T & T \\
T & F & T & T & T & T & T & T \\
T & T & F & T & F & F & T & T \\
T & T & T & T & F & F & T & T \\
\bottomrule
\end{tabular}
\]

%======================
% (B)
%======================

\subsection*{(B) Truth Table}
\emph{Informal argument.} \begin{enumerate} \item To get a good grade $G$ it is necessary to attend lectures and do readings: $G\to(L\land R)$. \item I attend lectures and do readings: $L\land R$. \end{enumerate} Conclusion: $G$.
\textbf{Validity:} \emph{Invalid.}

\begin{proof}[Counterexample]
Take the valuation $G=\F$, $L=\T$, $R=\T$. Then $G\to(L\land R)$ is true (antecedent false) and $L\land R$ is true, but $G$ is false. Therefore the premises can all be true while the conclusion is false; hence the argument is invalid. Intuitively the premise states ``$L\land R$ is \emph{necessary} for $G$'', which does not imply $L\land R$ is sufficient for $G$.
\end{proof}

\[
\begin{tabular}{ccc c c c c}
\toprule
G & L & R & $G\to(L\land R)$ & $L\land R$ & \text{All Premises?} & G \\
\midrule
F & F & F & T & F & F & F \\
F & F & T & T & F & F & F \\
F & T & F & T & F & F & F \\
F & T & T & T & T & T & F \\
T & F & F & F & F & F & T \\
T & F & T & F & F & F & T \\
T & T & F & F & F & F & T \\
T & T & T & T & T & T & T \\
\bottomrule
\end{tabular}
\]

%======================
% (C)
%======================
\subsection*{(C) Truth Table}
\emph{Informal argument.} \begin{enumerate} \item $((L\land R\land E)\lor K)\to G$. \item $L$, $R$, $E$. \end{enumerate} Conclusion: $G$.

\textbf{Validity:} \emph{Valid.}

\begin{enumerate}
  \item $((L\land R\land E)\lor K)\to G$ \hfill Premise
  \item $L$ \hfill Premise
  \item $R$ \hfill Premise
  \item $E$ \hfill Premise
  \item From 2--4 infer $L\land R\land E$. \hfill $\land$-Introduction
  \item From 5 infer $(L\land R\land E)\lor K$. \hfill $\lor$-Introduction
  \item From 1 and 6 infer $G$. \hfill Modus Ponens
\end{enumerate}
Thus $G$ follows; the argument is valid.

\[
\begin{tabular}{ccccc c c c c}
\toprule
L & R & E & K & G & $(L\land R\land E)\lor K$ & \text{Premise1: antecedent$\to G$} & \text{All Premises?} & G \\
\midrule
F & F & F & F & F & F & T & F & F \\
F & F & F & F & T & F & T & F & T \\
F & F & F & T & F & T & F & F & F \\
F & F & F & T & T & T & T & F & T \\
F & F & T & F & F & F & T & F & F \\
F & F & T & F & T & F & T & F & T \\
F & F & T & T & F & T & F & F & F \\
F & F & T & T & T & T & T & F & T \\
F & T & F & F & F & F & T & F & F \\
F & T & F & F & T & F & T & F & T \\
F & T & F & T & F & T & F & F & F \\
F & T & F & T & T & T & T & F & T \\
F & T & T & F & F & F & T & F & F \\
F & T & T & F & T & F & T & F & T \\
F & T & T & T & F & T & F & F & F \\
F & T & T & T & T & T & T & F & T \\
T & F & F & F & F & F & T & F & F \\
T & F & F & F & T & F & T & F & T \\
T & F & F & T & F & T & F & F & F \\
T & F & F & T & T & T & T & F & T \\
T & F & T & F & F & F & T & F & F \\
T & F & T & F & T & F & T & F & T \\
T & F & T & T & F & T & F & F & F \\
T & F & T & T & T & T & T & F & T \\
T & T & F & F & F & F & T & F & F \\
T & T & F & F & T & F & T & F & T \\
T & T & F & T & F & T & F & F & F \\
T & T & F & T & T & T & T & F & T \\
T & T & T & F & F & T & F & F & F \\
T & T & T & F & T & T & T & T & T \\
T & T & T & T & F & T & F & F & F \\
T & T & T & T & T & T & T & T & T \\
\bottomrule
\end{tabular}
\]

%======================
% (D)
%======================
\subsection*{(D) Truth Table}
\emph{Informal argument.} \begin{enumerate} \item $(L\to\neg B)\lor(K\to\neg C)$. \item $B\lor C$. \end{enumerate}
\textbf{Validity:} \emph{Invalid.}

\begin{proof}[Counterexample]
Set
\[
(B,C,L,K)=(\T,\F,\T,\T).
\]
Evaluate:
\[
L\to\neg B = \T\to\F = \F,\qquad K\to\neg C=\T\to\T=\T,
\]
so the disjunction $(L\to\neg B)\lor(K\to\neg C)$ is $\T$. Also $B\lor C=\T$. But the conclusion
\[
\neg L\lor\neg K=\neg\T\lor\neg\T=\F
\]
is false. Therefore premises true and conclusion false, so the argument is invalid.
\end{proof}

\[
\begin{tabular}{cccc c c c c c}
\toprule
L & K & B & C & $L\to \neg B$ & $K\to \neg C$ & $(L\to\neg B)\lor(K\to\neg C)$ & $B\lor C$ & \neg L\lor\neg K \\
\midrule
F & F & F & F & T & T & T & F & T \\
F & F & F & T & T & T & T & T & T \\
F & F & T & F & T & T & T & T & T \\
F & F & T & T & T & T & T & T & T \\
F & T & F & F & T & T & T & F & T \\
F & T & F & T & T & F & T & T & T \\
F & T & T & F & T & T & T & T & T \\
F & T & T & T & T & F & T & T & T \\
T & F & F & F & F & T & T & F & T \\
T & F & F & T & F & T & T & T & T \\
T & F & T & F & F & T & F & T & T \\
T & F & T & T & F & T & T & T & T \\
T & T & F & F & F & T & T & F & F \\
T & T & F & T & F & F & F & T & F \\
T & T & T & F & F & T & F & T & F \\
T & T & T & T & F & F & F & T & F \\
\bottomrule
\end{tabular}
\]

%======================
% (E)
%======================
\subsection*{(E) Truth Table}
\emph{Informal argument.} \begin{enumerate} \item $L\to\neg B$. \item $K\to\neg C$. \item $B\lor C$. \end{enumerate} Conclusion: $\neg L\lor\neg K$.
\textbf{Validity:} \emph{Valid.}

\begin{enumerate}
  \item $L\to\neg B$ \hfill Premise
  \item $K\to\neg C$ \hfill Premise
  \item $B\lor C$ \hfill Premise
  \item Assume, for reductio, $L\land K$.
  \begin{enumerate}
    \item From assumption infer $L$ and $K$.
    \item From 1 and $L$ infer $\neg B$.
    \item From 2 and $K$ infer $\neg C$.
    \item From 3 and $\neg B,\neg C$ conclude contradiction, since $B\lor C$ together with $\neg B\land\neg C$ is impossible.
  \end{enumerate}
  \item Hence $L\land K$ leads to contradiction; therefore $\neg(L\land K)$ holds.
  \item By De Morgan, $\neg(L\land K)\equiv \neg L\lor\neg K$, which is the desired conclusion.
\end{enumerate}
This completes the proof; the argument is valid.

\[
\begin{tabular}{cccc c c c c}
\toprule
L & K & B & C & $L\to\neg B$ & $K\to\neg C$ & $B\lor C$ & \neg L\lor\neg K \\
\midrule
F & F & F & F & T & T & F & T \\
F & F & F & T & T & T & T & T \\
F & F & T & F & T & T & T & T \\
F & F & T & T & T & T & T & T \\
F & T & F & F & T & T & F & T \\
F & T & F & T & T & F & T & T \\
F & T & T & F & T & T & T & T \\
F & T & T & T & T & F & T & T \\
T & F & F & F & F & T & F & T \\
T & F & F & T & F & T & T & T \\
T & F & T & F & F & T & T & T \\
T & F & T & T & F & T & T & T \\
T & T & F & F & F & T & F & F \\
T & T & F & T & F & F & T & F \\
T & T & T & F & F & T & T & F \\
T & T & T & T & F & F & T & F \\
\bottomrule
\end{tabular}
\]
\end{document}
