\documentclass[12pt]{article}

%============================================================
% PACKAGES
%============================================================
\usepackage{amsmath, amssymb, amsthm}
\usepackage{enumitem}
\usepackage{geometry}
\geometry{margin=1in}
\usepackage{titlesec}

%============================================================
% CUSTOM ENVIRONMENTS
%============================================================
\newtheorem{theorem}{Theorem}
\newtheorem{lemma}{Lemma}
\newtheorem{proposition}{Proposition}
\newtheorem{corollary}{Corollary}
\theoremstyle{definition}
\newtheorem{definition}{Definition}
\newtheorem{example}{Example}
\newtheorem{exercise}{Exercise}
\theoremstyle{remark}
\newtheorem*{remark}{Remark}

\newenvironment{proofsketch}{\paragraph{Proof Sketch.}}{\hfill$\square$}

%============================================================
% DOCUMENT
%============================================================
\begin{document}

\title{Logic and Proof Practice: Validity and Proof Strategy}
\author{Gosha Polstvin}
\date{\today}
\maketitle

\tableofcontents
\bigskip

%============================================================
\section{Argument Validity and Counterexamples}

\subsection*{1.1 Birds, Insects, and Flying Species (Valid)}

\textbf{Given:}
\[
\forall x(B(x) \rightarrow \exists y(I(y) \wedge E(x,y))), \quad 
\forall y(I(y) \rightarrow F(y)).
\]
\textbf{To Prove:}
\[
\forall x(B(x) \rightarrow \exists y(F(y) \wedge E(x,y))).
\]

\begin{proofsketch}
Let \(a\) be arbitrary such that \(B(a)\).
From the first premise, \(\exists y(I(y) \wedge E(a,y))\).
Instantiate to \(I(c) \wedge E(a,c)\).
By the second premise, \(I(c) \rightarrow F(c)\), so \(F(c)\) holds.
Hence \(F(c) \wedge E(a,c)\), therefore \(\exists y(F(y) \wedge E(a,y))\).
By universal generalization, 
\(\forall x(B(x) \rightarrow \exists y(F(y) \wedge E(x,y)))\).
\end{proofsketch}

\noindent\textbf{Conclusion:} Argument is \textbf{valid.}

%------------------------------------------------------------
\subsection*{1.2 Smart People, Money, and Big Houses (Invalid)}

\textbf{Premises:}
\[
\exists x(S(x) \wedge M(x)), \quad \exists x(M(x) \wedge H(x)).
\]
\textbf{Conclusion:}
\[
\exists x(S(x) \wedge H(x)).
\]

\begin{proofsketch}
Consider the domain \(\{a,b\}\).
Let \(S(a)\) and \(M(a)\) be true, but \(H(a)\) false.
Let \(M(b)\) and \(H(b)\) be true, but \(S(b)\) false.
Then:
\begin{itemize}[noitemsep]
    \item \(\exists x(S(x)\wedge M(x))\) true for \(x=a\);
    \item \(\exists x(M(x)\wedge H(x))\) true for \(x=b\);
    \item but \(\exists x(S(x)\wedge H(x))\) is false.
\end{itemize}
Thus, premises true and conclusion false.
\end{proofsketch}

\noindent\textbf{Conclusion:} Argument is \textbf{invalid.}

%------------------------------------------------------------
\subsection*{1.3 Knowing, Singing, and Liking Songs (Valid)}

\textbf{Given:}
\begin{align*}
1.\quad & \forall x(P(x) \rightarrow \exists y(S(y) \wedge K(x,y)))\\
2.\quad & \forall y(S(y) \rightarrow \exists z(P(z) \wedge \text{Sing}(z,y)))\\
3.\quad & \forall z \forall y((P(z) \wedge S(y) \wedge \text{Sing}(z,y)) \rightarrow L(z,y))
\end{align*}

\textbf{To Prove:}
\[
\forall x(P(x) \rightarrow \exists y(S(y) \wedge K(x,y) \wedge \exists z(P(z)\wedge L(z,y)))).
\]

\begin{proofsketch}
Let \(a\) be an arbitrary person, \(P(a)\).  
From (1), there exists a song \(s\) such that \(S(s)\wedge K(a,s)\).
From (2), since \(S(s)\), there exists a person \(t\) such that \(P(t)\wedge \text{Sing}(t,s)\).
From (3), using \(P(t)\wedge S(s)\wedge \text{Sing}(t,s)\), infer \(L(t,s)\).
Hence \(S(s)\wedge K(a,s)\wedge \exists z(P(z)\wedge L(z,s))\).
Existentially generalize and universalize to reach the desired conclusion.
\end{proofsketch}

\noindent\textbf{Conclusion:} Argument is \textbf{valid.}

%============================================================
\section{Choosing Proof Strategies}

\subsection*{2.1 If $x+y+z$ is odd, then at least one of $x,y,z$ is odd}

\begin{proofsketch}[Strategy: Contrapositive]
Contrapositive: If \(x,y,z\) are all even, then \(x+y+z\) is even.

If each is even, \(x=2a, y=2b, z=2c\) for integers \(a,b,c\).  
Then \(x+y+z=2(a+b+c)\), which is even.

Hence, if \(x+y+z\) is odd, not all of \(x,y,z\) can be even.
Therefore, at least one is odd.
\end{proofsketch}

%------------------------------------------------------------
\subsection*{2.2 $n$ even $\iff$ $7n+4$ even}

\begin{proofsketch}[Strategy: Biconditional direct proof]
$(\Rightarrow)$ If \(n\) is even, \(n=2k\). Then \(7n+4=14k+4=2(7k+2)\), even.\\
$(\Leftarrow)$ If \(7n+4\) is even, then \(7n\) is even.  
Since 7 is odd, dividing by 7 preserves evenness, so \(n\) is even.
\end{proofsketch}

%------------------------------------------------------------
\subsection*{2.3 If $A\cap B\neq \varnothing$ and $A\subseteq C$, then $B\cap C\neq \varnothing$}

\begin{proofsketch}[Strategy: Element-chase]
Since \(A\cap B\neq \varnothing\), there exists \(x\in A\cap B\).  
Thus \(x\in A\) and \(x\in B\).  
Given \(A\subseteq C\), we have \(x\in C\).  
Hence \(x\in B\cap C\), so \(B\cap C\neq \varnothing\).
\end{proofsketch}

\end{document}

